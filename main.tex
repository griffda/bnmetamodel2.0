\documentclass[lettersize,journal]{IEEEtran}
\usepackage{amsmath,amsfonts}
\usepackage{algorithmic}
\usepackage{array}
\usepackage{subcaption}
\usepackage{textcomp}
\usepackage{stfloats}
\usepackage{url}
\usepackage{verbatim}
\usepackage{booktabs}
\usepackage{multirow}
\usepackage{graphicx}
\usepackage{colortbl}
\usepackage[table,xcdraw]{xcolor}
\usepackage{tikz}
\usepackage{titlesec}
\usepackage{authblk}
\usepackage{marvosym}
\usetikzlibrary{positioning, calc}
\usetikzlibrary{arrows}
\newcolumntype{P}[1]{>{\raggedright\arraybackslash}p{#1}}

\hyphenation{op-tical net-works semi-conduc-tor IEEE-Xplore}
\def\BibTeX{{\rm B\kern-.05em{\sc i\kern-.025em b}\kern-.08em
    T\kern-.1667em\lower.7ex\hbox{E}\kern-.125emX}}
\usepackage{balance}
\begin{document}
\title{bnmetamodel 2.0}



\author[1]{T. Griffiths\thanks{\textsuperscript{\Cross}Corresponding author: t.griffiths20@imperial.ac.uk}}
\author[2]{Z. Xuereb Conti}
\author[1]{M. Bluck}

\affil[1]{Department of Mechanical Engineering, Imperial College London, UK}
\affil[2]{Data-Centric Engineering / TRIC:DT, The Alan Turing Institute, UK}
\vspace{-15pt}

\maketitle

\begin{abstract}

\end{abstract}

\begin{IEEEkeywords}
Fusion power, metamodels, surrogate modelling, fusion commercialisation, machine learning, fusion economics, energy, Bayesian Networks
\end{IEEEkeywords}
\vspace{-2ex}

\section{Introduction}
This paper will include topics not discussed in first ML paper. Hyperparameter tuning, model selection, and model evaluation. Another case study will be included, this will use data from another fusion developer to help predict power plant economics. Addition of new data or information will be included. Addition of new node (child or parent) will be included. Addition of soft evidence will be included.

Commercial-scale fusion power promises a future of reliable baseload, reduced carbon emissions, and enhanced energy security. Techno-economic modelling of future fusion power plants, faces challenges due to uncertain and imprecise costing models. To reason over uncertainty, efforts to estimate fusion power economics have been previously demonstrated using probabilistic methods[cite: Griffiths2024].

Despite these challenges, it is crucial for the fusion community to continue research efforts to predict both technical and economic performance metrics. The lack of such initiatives could hinder investment attraction, roadmap target achievement, and the promotion of additional investment opportunities. One way to address uncertainty is through the use of statistical methods, such as sensitivity analyses, during the modelling process. This allows decision-makers to evaluate key modelling variables more thoroughly and understand where improvements can be made to enhance the reliability of predictions.

Computational models offer a valuable tool for developing understanding in areas lacking experimental data. They provide the flexibility to simulate various scenarios and conditions rapidly, allowing for quick iteration and parameter modification. This enables swift exploration and optimisation of designs without the need for physical modifications or repeated experiments. Given the complexity of fusion engineering systems, computational models can effectively handle numerous variables and interactions, facilitating the analysis of large-scale systems with intricate behaviours.

In this follow-up study, we apply the same computational modelling technique to a new case study dataset, incorporating adaptations and improvements to the method. This paper aims to further the understanding of fusion power economics and contribute to the ongoing efforts in fusion research and development.

This study represents a continuation of the surrogate modelling proof of concept study presented by Griffiths et al[cite: Griffiths2024]. It presents a novel method for dealing with uncertainty in fusion research, away from traditional techniques. The objective of surrogate modelling is to develop a streamlined and consequently quicker model that replicates the desired output of a more complex model, considering its inputs and parameters. In this context, a Bayesian Network serves as a surrogate for the a fusion systems code [INSERT PyTOK REF?] to predict the economics of fusion power plants under uncertain data, specifically focusing on Spherical Tokamaks (STs). 

\section{Bayesian Networks}\label{sec:BNs}


\subsection{Literature review}

\section{Methodology}\label{sec:PROCESS}


\section{Results}\label{sec: Results} 



\section{Discussion}\label{sec: Discussion}


\section{Conclusion and Further Work}\label{sec:conc}


\section{Acknowledgments}
This research was supported by the EPSRC (Engineering and Physical Sciences Research Council, UK) Nuclear Energy Futures Centre for Doctoral. Training in Nuclear Energy (NEF CDT).  Other research studies under the NEF CDT involving Thomas Griffiths are supported in part by Tokamak Energy Ltd, UK. Views and opinions expressed are however those of the author(s) only a do not necessarily reflect those of Tokamak Energy Ltd.

\scriptsize\bibliographystyle{vancouver}
\scriptsize\bibliography{references.bib}

\end{document}